% Options for packages loaded elsewhere
% Options for packages loaded elsewhere
\PassOptionsToPackage{unicode}{hyperref}
\PassOptionsToPackage{hyphens}{url}
\PassOptionsToPackage{dvipsnames,svgnames,x11names}{xcolor}
%
\documentclass[
  letterpaper,
  DIV=11,
  numbers=noendperiod]{scrartcl}
\usepackage{xcolor}
\usepackage[margin=0.75in]{geometry}
\usepackage{amsmath,amssymb}
\setcounter{secnumdepth}{-\maxdimen} % remove section numbering
\usepackage{iftex}
\ifPDFTeX
  \usepackage[T1]{fontenc}
  \usepackage[utf8]{inputenc}
  \usepackage{textcomp} % provide euro and other symbols
\else % if luatex or xetex
  \usepackage{unicode-math} % this also loads fontspec
  \defaultfontfeatures{Scale=MatchLowercase}
  \defaultfontfeatures[\rmfamily]{Ligatures=TeX,Scale=1}
\fi
\usepackage{lmodern}
\ifPDFTeX\else
  % xetex/luatex font selection
\fi
% Use upquote if available, for straight quotes in verbatim environments
\IfFileExists{upquote.sty}{\usepackage{upquote}}{}
\IfFileExists{microtype.sty}{% use microtype if available
  \usepackage[]{microtype}
  \UseMicrotypeSet[protrusion]{basicmath} % disable protrusion for tt fonts
}{}
\makeatletter
\@ifundefined{KOMAClassName}{% if non-KOMA class
  \IfFileExists{parskip.sty}{%
    \usepackage{parskip}
  }{% else
    \setlength{\parindent}{0pt}
    \setlength{\parskip}{6pt plus 2pt minus 1pt}}
}{% if KOMA class
  \KOMAoptions{parskip=half}}
\makeatother
% Make \paragraph and \subparagraph free-standing
\makeatletter
\ifx\paragraph\undefined\else
  \let\oldparagraph\paragraph
  \renewcommand{\paragraph}{
    \@ifstar
      \xxxParagraphStar
      \xxxParagraphNoStar
  }
  \newcommand{\xxxParagraphStar}[1]{\oldparagraph*{#1}\mbox{}}
  \newcommand{\xxxParagraphNoStar}[1]{\oldparagraph{#1}\mbox{}}
\fi
\ifx\subparagraph\undefined\else
  \let\oldsubparagraph\subparagraph
  \renewcommand{\subparagraph}{
    \@ifstar
      \xxxSubParagraphStar
      \xxxSubParagraphNoStar
  }
  \newcommand{\xxxSubParagraphStar}[1]{\oldsubparagraph*{#1}\mbox{}}
  \newcommand{\xxxSubParagraphNoStar}[1]{\oldsubparagraph{#1}\mbox{}}
\fi
\makeatother


\usepackage{longtable,booktabs,array}
\usepackage{calc} % for calculating minipage widths
% Correct order of tables after \paragraph or \subparagraph
\usepackage{etoolbox}
\makeatletter
\patchcmd\longtable{\par}{\if@noskipsec\mbox{}\fi\par}{}{}
\makeatother
% Allow footnotes in longtable head/foot
\IfFileExists{footnotehyper.sty}{\usepackage{footnotehyper}}{\usepackage{footnote}}
\makesavenoteenv{longtable}
\usepackage{graphicx}
\makeatletter
\newsavebox\pandoc@box
\newcommand*\pandocbounded[1]{% scales image to fit in text height/width
  \sbox\pandoc@box{#1}%
  \Gscale@div\@tempa{\textheight}{\dimexpr\ht\pandoc@box+\dp\pandoc@box\relax}%
  \Gscale@div\@tempb{\linewidth}{\wd\pandoc@box}%
  \ifdim\@tempb\p@<\@tempa\p@\let\@tempa\@tempb\fi% select the smaller of both
  \ifdim\@tempa\p@<\p@\scalebox{\@tempa}{\usebox\pandoc@box}%
  \else\usebox{\pandoc@box}%
  \fi%
}
% Set default figure placement to htbp
\def\fps@figure{htbp}
\makeatother





\setlength{\emergencystretch}{3em} % prevent overfull lines

\providecommand{\tightlist}{%
  \setlength{\itemsep}{0pt}\setlength{\parskip}{0pt}}



 


\KOMAoption{captions}{tableheading}
\makeatletter
\@ifpackageloaded{tcolorbox}{}{\usepackage[skins,breakable]{tcolorbox}}
\@ifpackageloaded{fontawesome5}{}{\usepackage{fontawesome5}}
\definecolor{quarto-callout-color}{HTML}{909090}
\definecolor{quarto-callout-note-color}{HTML}{0758E5}
\definecolor{quarto-callout-important-color}{HTML}{CC1914}
\definecolor{quarto-callout-warning-color}{HTML}{EB9113}
\definecolor{quarto-callout-tip-color}{HTML}{00A047}
\definecolor{quarto-callout-caution-color}{HTML}{FC5300}
\definecolor{quarto-callout-color-frame}{HTML}{acacac}
\definecolor{quarto-callout-note-color-frame}{HTML}{4582ec}
\definecolor{quarto-callout-important-color-frame}{HTML}{d9534f}
\definecolor{quarto-callout-warning-color-frame}{HTML}{f0ad4e}
\definecolor{quarto-callout-tip-color-frame}{HTML}{02b875}
\definecolor{quarto-callout-caution-color-frame}{HTML}{fd7e14}
\makeatother
\makeatletter
\@ifpackageloaded{caption}{}{\usepackage{caption}}
\AtBeginDocument{%
\ifdefined\contentsname
  \renewcommand*\contentsname{Table of contents}
\else
  \newcommand\contentsname{Table of contents}
\fi
\ifdefined\listfigurename
  \renewcommand*\listfigurename{List of Figures}
\else
  \newcommand\listfigurename{List of Figures}
\fi
\ifdefined\listtablename
  \renewcommand*\listtablename{List of Tables}
\else
  \newcommand\listtablename{List of Tables}
\fi
\ifdefined\figurename
  \renewcommand*\figurename{Figure}
\else
  \newcommand\figurename{Figure}
\fi
\ifdefined\tablename
  \renewcommand*\tablename{Table}
\else
  \newcommand\tablename{Table}
\fi
}
\@ifpackageloaded{float}{}{\usepackage{float}}
\floatstyle{ruled}
\@ifundefined{c@chapter}{\newfloat{codelisting}{h}{lop}}{\newfloat{codelisting}{h}{lop}[chapter]}
\floatname{codelisting}{Listing}
\newcommand*\listoflistings{\listof{codelisting}{List of Listings}}
\makeatother
\makeatletter
\makeatother
\makeatletter
\@ifpackageloaded{caption}{}{\usepackage{caption}}
\@ifpackageloaded{subcaption}{}{\usepackage{subcaption}}
\makeatother
\usepackage{bookmark}
\IfFileExists{xurl.sty}{\usepackage{xurl}}{} % add URL line breaks if available
\urlstyle{same}
\hypersetup{
  pdftitle={Syllabus},
  pdfauthor={Dr.~Anna Rosen},
  colorlinks=true,
  linkcolor={blue},
  filecolor={Maroon},
  citecolor={Blue},
  urlcolor={Blue},
  pdfcreator={LaTeX via pandoc}}


\title{Syllabus}
\usepackage{etoolbox}
\makeatletter
\providecommand{\subtitle}[1]{% add subtitle to \maketitle
  \apptocmd{\@title}{\par {\large #1 \par}}{}{}
}
\makeatother
\subtitle{ASTR 201: Astronomy for Science Majors \textbar{} Spring 2026}
\author{Dr.~Anna Rosen}
\date{}
\begin{document}
\maketitle


\subsection{Course Information}\label{sec-course-info}

\begin{longtable}[]{@{}
  >{\raggedright\arraybackslash}p{(\linewidth - 2\tabcolsep) * \real{0.5000}}
  >{\raggedright\arraybackslash}p{(\linewidth - 2\tabcolsep) * \real{0.5000}}@{}}
\toprule\noalign{}
\endhead
\bottomrule\noalign{}
\endlastfoot
\textbf{Instructor} & Dr.~Anna Rosen \\
\textbf{Email} &
\href{mailto:alrosen@sdsu.edu}{\nolinkurl{alrosen@sdsu.edu}} \\
\textbf{Office} & Physics 239 \\
\textbf{Office Hours} & Fridays 11:00 am--12:00 pm (and by
appointment) \\
\textbf{Class meetings} & Tues/Thurs 12:30--1:45 pm \\
\textbf{Location} & LH 245 \\
\textbf{Course website} &
\textless{}https://astrobytes-edu.github.io/astr201-sp26\textgreater{}
(TBD) \\
\textbf{Platforms:} \emph{(where things happen)} &
\href{https://sdsu.instructure.com}{Canvas} \emph{(announcements,
submissions,
gradebook)}\href{https://astrobytes-edu.github.io/astr201-sp26}{Course
website} \emph{(course materials, handouts, assignments)}iClicker
\emph{(in-class engagement)} \\
\textbf{Virtual ASTR 201 AI Tutors:} \emph{(for studying, enhanced
learning)} & \textbf{ASTR 201 Notebook (NotebookLM)} \emph{(grounded in
course materials only)} \textbf{ASTR 201 Socratic Tutor (Custom GPT)}
\emph{(practice coach, not a solution engine)} \\
\end{longtable}

\subsection{Start Here}\label{start-here}

\begin{itemize}
\tightlist
\item
  \textbf{Where to find things:} check Canvas + the course website
  (links above).
\item
  \textbf{Weekly cadence:} homework due \textbf{Mon 11:59 pm}; grade
  memo due \textbf{Wed 11:59 pm}.
\item
  \textbf{Exams:} closed-note/closed-book; calculator allowed (no
  phones); formula sheet provided.
\item
  \textbf{Getting help:} see \hyperref[sec-getting-help]{Getting Help
  (Office Hours)}.
\end{itemize}

\subsubsection{Course Description}\label{course-description}

Directed toward students with a \emph{strong interest} in science and
mathematics. This course builds a foundational and quantitative
understanding of modern astronomy, spanning: how we infer physical
properties from light; gravity and orbital motion; stars and stellar
evolution; the interstellar medium and star/planet formation; galaxies
and dark matter; and cosmology.

\textbf{Prerequisite:} Satisfaction of the SDSU Mathematics/Quantitative
Reasoning Assessment requirement.

\textbf{Important:} This is a \textbf{quantitative} course: we'll use
math to build \textbf{physical understanding and intuition}, not rote
memorization or ``plug-and-chug.'' You'll practice interpreting
equations, stating assumptions, and sanity-checking results (units,
limits, scaling); as a required course for Astronomy pre-majors, ASTR
201 is also designed to prepare you for \textbf{upper-division Physics
and Astronomy} through scientific modeling, quantitative reasoning, and
clear scientific explanation.

\begin{tcolorbox}[enhanced jigsaw, leftrule=.75mm, toprule=.15mm, opacityback=0, opacitybacktitle=0.6, left=2mm, toptitle=1mm, coltitle=black, colbacktitle=quarto-callout-note-color!10!white, colframe=quarto-callout-note-color-frame, colback=white, bottomrule=.15mm, bottomtitle=1mm, arc=.35mm, breakable, titlerule=0mm, rightrule=.15mm, title={Core Course Throughline}]

\emph{Modern astronomy is astrophysics: \textbf{inferring physical
reality from constrained measurements.}}

We collect photons (and other messengers), translate limited
observations into physical quantities, build simplified models, quantify
uncertainty, and test whether those models can explain what we observe.

\end{tcolorbox}

\subsubsection{Required Materials}\label{required-materials}

\begin{itemize}
\item
  
\item
  \textbf{Required ed-tech:} iClicker (student access)
  \url{https://student.iclicker.com/} \emph{(Required for in-class
  engagement)}
\item
  \textbf{Other supplies:} A calculator (non-smartphone) for exams and
  reliable access to Canvas and the course website.
\end{itemize}

\subsection{Course Learning Outcomes}\label{sec-outcomes}

By the end of this course, students will be able to:

\begin{enumerate}
\def\labelenumi{\arabic{enumi}.}
\item
  \textbf{Infer} the physical properties of celestial objects from
  electromagnetic radiation by \textbf{analyzing wavelength, intensity,
  and spectral features}, and by \textbf{explicitly connecting observed
  signals to underlying physical processes}.
\item
  Apply fundamental physical principles---including gravity,
  conservation laws, and orbital mechanics---to \textbf{analyze and
  quantitatively justify} the motions and interactions of celestial
  bodies in the solar system and beyond.
\item
  \textbf{Construct and evaluate simplified physical models} of stars
  and planets using blackbody radiation, spectral lines, and Doppler
  shifts, and \textbf{assess the assumptions and limitations of those
  models}.
\item
  \textbf{Reason from initial conditions}---such as stellar mass and
  composition---to predict and explain stellar lifecycles, explicitly
  linking gravity, nuclear fusion, and energy balance to observed
  evolutionary outcomes.
\item
  \textbf{Infer} the structure, mass distribution, and evolution of
  galaxies from \textbf{observational evidence---such as rotation curves
  and luminosity profiles---and evaluate the necessity of dark matter}
  in explaining galactic dynamics.
\item
  \textbf{Analyze cosmological observations and scaling relations} to
  develop a physically grounded understanding of the origin, evolution,
  and large-scale structure of the Universe, including the roles of dark
  energy and cosmic expansion.
\item
  Demonstrate how astronomical observations are used to
  \textbf{construct, test, refine, and falsify physical models}, and
  \textbf{articulate how uncertainty, assumptions, and observational
  limits shape scientific conclusions}.
\item
  \textbf{Use quantitative reasoning and physical principles developed
  in astronomy} to critically evaluate scientific claims, identify
  disinformation, and assess real-world challenges involving energy,
  climate, and space-related science.
\end{enumerate}

\subsection{Grading \& Assessments}\label{sec-grading}

\begin{longtable}[]{@{}lr@{}}
\toprule\noalign{}
Component & Weight \\
\midrule\noalign{}
\endhead
\bottomrule\noalign{}
\endlastfoot
Scholarly Engagement (iClicker + in-class work) & 10\% \\
Homework + Grade Memos & 15\% \\
Growth Memos (3) & 10\% \\
Midterm Exam 1 & 15\% \\
Midterm Exam 2 & 15\% \\
Final Exam (cumulative) & 35\% \\
\end{longtable}

\subsubsection{Important Dates}\label{sec-important-dates}

\begin{itemize}
\tightlist
\item
  \textbf{Midterm Exam 1:} Thursday, {[}TBD{]}
\item
  \textbf{Midterm Exam 2:} Thursday, {[}TBD{]}
\item
  \textbf{Final Exam:} Thursday, May 7 --- 10:30 a.m.--12:30 p.m.
\end{itemize}

\textbf{Course schedule:} The tentative day-by-day schedule (topics,
readings, homework calendar) is on the course website:
\href{./schedule.qmd}{Schedule page (to be published)}

\subsubsection{Grading scale}\label{sec-grade-scale}

\begin{longtable}[]{@{}lll@{}}
\toprule\noalign{}
Letter grade & Percent range & Explanation \\
\midrule\noalign{}
\endhead
\bottomrule\noalign{}
\endlastfoot
A & 93--100\% & Outstanding \\
A- & 90--92\% & \\
B+ & 87--89\% & \\
B & 83--86\% & Praiseworthy \\
B- & 80--82\% & \\
C+ & 77--79\% & \\
C & 73--76\% & Average \\
C- & 70--72\% & \\
D+ & 67--69\% & \\
D & 63--66\% & Minimally Passing \\
D- & 60--62\% & \\
F & Below 60\% & Failure \\
\end{longtable}

\emph{These percentages represent guaranteed thresholds---earning the
stated percentage guarantees at least that grade. The instructor
reserves the right to adjust borderline grades upward based on effort,
improvement, and engagement.}

For SDSU's explanation of grades, see:
\url{https://registrar.sdsu.edu/faculty_staff/courses_grades/explanation-grades}

\begin{center}\rule{0.5\linewidth}{0.5pt}\end{center}

\subsubsection{Course Components}\label{sec-components}

Per SDSU policy, students are expected to spend at least 6 hours per
week (a minimum of 2 hours per course credit hour/unit) on coursework
outside of class, including reading, homework, and exam preparation for
this 3-unit course. (see
\url{https://catalog.sdsu.edu/content.php?catoid=10&navoid=924\#credit-hour-or-unit}).

\textbf{In this section:} - \hyperref[sec-engagement]{Scholarly
Engagement} - \hyperref[sec-homework]{Weekly Homework + Grade Memos} -
\hyperref[sec-growth-memos]{Growth Memos} - \hyperref[sec-exams]{Exams}

\paragraph{Scholarly Engagement (10\%)}\label{sec-engagement}

Scholarly engagement in this course means making your reasoning
\emph{visible}. In practice, that looks like explaining what an equation
means in words, carrying units through your work, stating assumptions
out loud (even when they feel ``obvious''), and being willing to test
and revise a claim when evidence disagrees. These habits are what turn
astronomy from memorizing facts into doing inference from data---and
they're also what make your learning robust in an era where AI can
generate fluent text but can't replace \emph{your} physical judgment.

Engagement is measured via \texttt{iClicker} responses plus observable
weekly behaviors.

\begin{longtable}[]{@{}
  >{\raggedright\arraybackslash}p{(\linewidth - 2\tabcolsep) * \real{0.2500}}
  >{\raggedright\arraybackslash}p{(\linewidth - 2\tabcolsep) * \real{0.7500}}@{}}
\toprule\noalign{}
\begin{minipage}[b]{\linewidth}\raggedright
\textbf{Instructor Score}
\end{minipage} & \begin{minipage}[b]{\linewidth}\raggedright
\textbf{Observable weekly behaviors}
\end{minipage} \\
\midrule\noalign{}
\endhead
\bottomrule\noalign{}
\endlastfoot
5/5 & Prepared (specific questions on readings/notes), engaged in
iClicker + activities, contributes to discussion, helps peers reason
through problems, uses evidence/units/assumptions in explanations \\
4/5 & Prepared, steady effort in activities, contributes occasionally,
collaborates respectfully, asks questions when stuck \\
3/5 & Inconsistent prep, participates when prompted, limited
collaboration/discussion contribution \\
2/5 & Frequently unprepared or disengaged, minimal contribution to
activities, distracts self/others \\
1/5 & Rare attendance/engagement, unprepared when present, does not
participate in activities \\
0/5 & Habitually absent, no engagement \\
\end{longtable}

\textbf{Note:} This rubric is a guideline. Your score reflects overall
scholarly contribution, not a single ``style'' of being present.

\textbf{Classroom Norms \& Attendance (Professional Scientific
Community):} We are all members of a professional academic community.
Treat classmates and the instructor with respect; spirited, courteous
debate about scientific ideas is welcome, but disrespectful behavior is
not. Because active reasoning is the point of being in the room, phones
and other non-course device use during class are not
permitted---off-task use is distracting and will negatively affect your
Scholarly Engagement score.

Formal attendance is not taken, but you cannot earn Scholarly Engagement
credit if you are not present and participating. Please arrive on time;
repeated tardiness is disruptive and will negatively affect your
Scholarly Engagement score.

\paragraph{Weekly Homework + Grade Memos (15\%)}\label{sec-homework}

Homework builds quantitative fluency and model-based reasoning. Expect
multi-step problems and conceptual questions where \textbf{units,
assumptions, and physical interpretation} matter as much as the final
number. The purpose of homework is \textbf{exam preparation and
skill-building} through consistent, high-quality practice --- not busy
work.

\textbf{Submission workflow (two-stage):}

\begin{enumerate}
\def\labelenumi{\arabic{enumi}.}
\tightlist
\item
  \textbf{Homework Solutions --- due Monday 11:59 pm PT (Canvas)}

  \begin{itemize}
  \tightlist
  \item
    Must be uploaded as \textbf{one single, readable PDF} (not a photo
    dump).
  \item
    Organize clearly; show work; label final answers.
  \item
    \textbf{No late submissions:} Instructor solutions will be posted
    \textbf{Tuesday morning}, so late work cannot be accepted.
  \item
    \textbf{Lowest homework score will be dropped} (to cover one
    off-week or emergency).
  \end{itemize}
\item
  \textbf{Self-Assessment + Reflection (``Grade Memo'') --- due
  Wednesday 11:59 pm PT (Canvas)}

  \begin{itemize}
  \tightlist
  \item
    You will \textbf{self-assess (self-grade)} your work using the
    \textbf{detailed homework rubric} and the posted solutions/guidance.
  \item
    You will submit a brief \textbf{grade memo (\textasciitilde250--400
    words)} that includes:

    \begin{itemize}
    \tightlist
    \item
      what you got right (and why),
    \item
      what broke (and where),
    \item
      what you learned,
    \item
      what you will do differently next time.
    \end{itemize}
  \item
    Your grade memo must also include:

    \begin{itemize}
    \tightlist
    \item
      a \textbf{per-problem self-rating} (1--5) with brief
      justification, and
    \item
      \textbf{AI and collaboration disclosure} (even if ``none'').
    \end{itemize}
  \item
    \textbf{Vague memos} (e.g., ``I need to study more'') will not earn
    full credit unless they include a \textbf{specific error diagnosis}
    and a \textbf{concrete next-step habit}.
  \end{itemize}
\end{enumerate}

\textbf{How your homework is graded (Instructor 0--5 score):} I will
evaluate your \emph{combined submission} (Monday solutions + Wednesday
grade memo) and assign an overall score from \textbf{0--5}. Homework is
graded primarily on \textbf{completion, professionalism, and learning
behaviors}, not just final correctness. ``Professionalism'' here means
your work is readable, logically organized, shows steps and units, and
reflects honest effort. Your grade memo is graded on the quality of your
self-assessment, reflection, and evidence of growth.

\begin{longtable}[]{@{}
  >{\raggedright\arraybackslash}p{(\linewidth - 2\tabcolsep) * \real{0.2500}}
  >{\raggedright\arraybackslash}p{(\linewidth - 2\tabcolsep) * \real{0.7500}}@{}}
\toprule\noalign{}
\begin{minipage}[b]{\linewidth}\raggedright
Instructor Score
\end{minipage} & \begin{minipage}[b]{\linewidth}\raggedright
What it means
\end{minipage} \\
\midrule\noalign{}
\endhead
\bottomrule\noalign{}
\endlastfoot
\textbf{5} & Complete, readable work with steps/units shown; thoughtful
self-assessment aligned with evidence; specific correction + diagnosis +
next-step habit; disclosures complete and honest. \\
\textbf{4} & Strong work with minor gaps (some missing steps/units or a
thinner memo), but clear effort and genuine reflection. \\
\textbf{3} & Adequate but inconsistent: missing reasoning in multiple
places and/or superficial reflection (e.g., ``need to study more''
without diagnosis). \\
\textbf{2} & Partial/low-quality: many incomplete or unclear solutions
and/or missing required memo elements. \\
\textbf{1} & Minimal effort; little usable work or reflection. \\
\textbf{0} & Not submitted, or missing required disclosures, or academic
integrity violation. \\
\end{longtable}

\textbf{Bottom line:} correct answers matter, but \textbf{visible
reasoning + honest reflection + improvement over time} matter more. This
is how you build exam-ready mastery.

\paragraph{Growth Memos (10\%)}\label{sec-growth-memos}

Growth Memos are short, structured reflections (``exam wrappers'') that
build \textbf{metacognition} --- thinking about how you learn so you can
study more effectively and improve exam performance. \textbf{Learning
science shows that structured reflection after feedback helps students
plan, monitor, and adjust their learning strategies.} Write these in an
\textbf{informal, memo-like voice} (not an essay). Honesty and ``being
real'' matter more than polished prose.

You will submit \textbf{three} Growth Memos:

\begin{itemize}
\tightlist
\item
  \textbf{GM1} --- after Midterm 1 (Due: TBD)
\item
  \textbf{GM2} --- after Midterm 2 (Due: TBD)
\item
  \textbf{GM3} --- pre-final synthesis memo (Due: TBD)
\end{itemize}

\textbf{Format:} You must use the \textbf{provided Growth Memo Template}
(posted on the course website/Canvas). \textbf{Length target:}
\textasciitilde{}\textbf{400--600 words}.

Each memo must include: evidence from your work, 2--3 recurring patterns
+ likely causes, 2--3 concrete next-step habits (``When I\ldots, I
will\ldots{}''), a verification plan, and a brief learning + AI
reflection (even if ``no AI used'').

\subparagraph{Grading (Instructor 0--5;
averaged)}\label{grading-instructor-05-averaged}

Each Growth Memo is scored \textbf{0--5} using the rubric below. Your
Growth Memos grade (\textbf{10\%}) is the \textbf{average} of your three
scores. Missing a memo earns a \textbf{0} for that memo.

\begin{longtable}[]{@{}
  >{\raggedright\arraybackslash}p{(\linewidth - 2\tabcolsep) * \real{0.2000}}
  >{\raggedright\arraybackslash}p{(\linewidth - 2\tabcolsep) * \real{0.8000}}@{}}
\toprule\noalign{}
\begin{minipage}[b]{\linewidth}\raggedright
Instructor Score
\end{minipage} & \begin{minipage}[b]{\linewidth}\raggedright
What it looks like
\end{minipage} \\
\midrule\noalign{}
\endhead
\bottomrule\noalign{}
\endlastfoot
\textbf{5} & Specific, evidence-based patterns; concrete habit plan;
clear verification plan; honest reflection. \\
\textbf{4} & Strong overall, but one element is thin (evidence,
specificity, or verification). \\
\textbf{3} & Some insight, but too general (``study more'') and/or lacks
actionable next steps. \\
\textbf{2--1} & Minimal effort; mostly summary; little diagnosis or
plan. \\
\textbf{0} & Not submitted or academic integrity violation. \\
\end{longtable}

\paragraph{Exams (65\% total)}\label{sec-exams}

There are two midterm exams (15\% each) and one \textbf{cumulative}
final exam (35\%). Exams are \textbf{closed-note and closed-book} and
emphasize \textbf{first-principles reasoning}: you'll be asked to
connect observables to physical models, justify steps from core ideas
(e.g., conservation laws, gravity, radiation/scaling arguments), and
explain what your result \emph{means} physically (units, assumptions,
limiting-case checks) --- not just compute a final value. You may use a
\textbf{calculator} (no phones/smartwatches). A \textbf{course formula
sheet will be provided with the exam}.

\subparagraph{Missed Exams}\label{sec-missed-exams}

There are \textbf{no make-up exams} in this course. The only exception
is a \textbf{serious, documented emergency} (e.g., hospitalization or a
verified family emergency). In that case, a make-up may be offered
\textbf{only at the instructor's discretion} and may use a
\textbf{different format} than the original exam. Travel, work
obligations, scheduling conflicts, technical issues, or forgetting an
exam do \textbf{not} qualify. An unexcused absence earns a
\textbf{zero}.

\emph{Final exam note:} SDSU's policy states that \textbf{no final exam
may be given to individual students before the regular scheduled time};
if it is impossible to take the final as scheduled, arrangements must be
made with the instructor (typically via an \textbf{Incomplete/deferred
final} process or \textbf{withdrawal} from the course, when applicable).
(\href{https://registrar.sdsu.edu/calendars/finals/fall-2025?utm_source=chatgpt.com}{SDSU
Registrar})

\subsection{Academic Integrity \& Generative AI
Policy}\label{sec-ai-policy}

All submitted work in this course must reflect \textbf{your own
understanding}.

Collaboration is encouraged to discuss concepts and strategies. However,
what you submit must be written independently and must represent your
unique reasoning. Copying solutions (from classmates, the internet, or
solution manuals) is not permitted.

\subsubsection{Generative AI (ChatGPT, Gemini, Claude,
etc.)}\label{sec-generative-ai}

\begin{tcolorbox}[enhanced jigsaw, leftrule=.75mm, toprule=.15mm, opacityback=0, opacitybacktitle=0.6, left=2mm, toptitle=1mm, coltitle=black, colbacktitle=quarto-callout-warning-color!10!white, colframe=quarto-callout-warning-color-frame, colback=white, bottomrule=.15mm, bottomtitle=1mm, arc=.35mm, breakable, titlerule=0mm, rightrule=.15mm, title=\textcolor{quarto-callout-warning-color}{\faExclamationTriangle}\hspace{0.5em}{AI Policy}]

\textbf{Allowed (study support):}

\begin{itemize}
\tightlist
\item
  Clarifying your own notes or assigned readings
\item
  Generating practice questions (not answers to assigned problems)
\item
  Explaining concepts at a different level \emph{for studying}
\end{itemize}

\textbf{Not allowed (graded work):}

\begin{itemize}
\tightlist
\item
  Generating or rewriting homework solutions, derivations, or
  explanations you submit
\item
  Submitting AI-generated reasoning you cannot reproduce on your own
\end{itemize}

\textbf{Always disclose AI use} in your grade memos, even if the use was
allowed.

\end{tcolorbox}

Generative AI can produce fluent explanations and plausible-looking
solutions that are subtly wrong. In a quantitative, model-based course,
``sounds right'' is not a standard of truth. If you don't have the
physics and quantitative discipline to verify an AI output, you risk
learning the wrong thing confidently.

To support effective studying, I may provide course-specific tools
grounded in \textbf{course materials only} (lecture notes, textbook,
vetted sources), such as a NotebookLM notebook and/or a Socratic
practice tutor. These are intended as study aids---not solution engines.

Violations of academic integrity will be handled according to university
procedures and may be reported to the Center for Student Rights and
Responsibilities. More information:
\url{https://sacd.sdsu.edu/student-rights/academic-dishonesty/cheating-and-plagiarism}

\subsection{Course Materials (Sharing Policy)}\label{sec-materials}

Lecture notes, slides, homework assignments, and exams are the
intellectual property of the instructor. You may use course materials
for your own educational purposes in this course, but you may not
reproduce, distribute, share, or post them on any public platform (e.g.,
Chegg, Course Hero, Discord servers, public GitHub repos) without
written permission.

Unauthorized recording or redistribution of class sessions or office
hours is prohibited.

\subsection{Communication \& Getting Help}\label{sec-communication}

All course-related communication should be sent through Canvas messaging
when possible. Include ``ASTR 201'' in the subject line. I aim to
respond within 24--48 hours on weekdays. \textbf{Before emailing, check
this syllabus and recent Canvas announcements.}

If you encounter difficulties with the assignments or find yourself
falling behind, please seek help immediately by:

\begin{itemize}
\tightlist
\item
  Posting your questions on the course \emph{Canvas} discussion page.
\item
  Speaking with the instructor during class or office hours (preferred).
\item
  Collaborating with classmates (while ensuring your final submission
  reflects your own understanding).
\item
  Attending SDSU's \emph{Astronomy Help Room} for free tutoring
  (schedule posted on the Canvas course page).
\end{itemize}

\subsubsection{Getting Help (Office Hours)}\label{sec-getting-help}

Office hours are for learning---bring what you've tried (notes,
attempts, questions) and we'll work from there. I'll often guide you by
asking questions so you build your own solution skills. You're welcome
to come even if you're not sure what to ask. Tips:
\href{https://lsc.cornell.edu/consult-with/how-to-use-office-hours/}{\emph{How
to Use Office Hours} (Cornell)}.

\subsection{Fostering a Growth Mindset}\label{sec-growth-mindset}

\begin{tcolorbox}[enhanced jigsaw, leftrule=.75mm, toprule=.15mm, opacityback=0, opacitybacktitle=0.6, left=2mm, toptitle=1mm, coltitle=black, colbacktitle=quarto-callout-tip-color!10!white, colframe=quarto-callout-tip-color-frame, colback=white, bottomrule=.15mm, bottomtitle=1mm, arc=.35mm, breakable, titlerule=0mm, rightrule=.15mm, title=\textcolor{quarto-callout-tip-color}{\faLightbulb}\hspace{0.5em}{Growth Mindset}]

A \textbf{growth mindset} is the belief that intelligence, abilities,
and talents can be developed through effort and persistence---not fixed
traits you're born with.

In ASTR 201, you'll encounter challenging concepts that may initially
seem overwhelming. \textbf{This is normal and expected: it means you're
learning.}

Trust the process, embrace the challenge, and discover that you're
capable of doing hard things.

\end{tcolorbox}

\subsection{Diversity and Inclusivity Statement}\label{sec-inclusion}

I consider this classroom to be a place where you will be treated with
respect, and I welcome individuals of all ages, backgrounds, beliefs,
ethnicities, genders, gender identities, gender expressions, national
origins, religious affiliations, sexual orientations, ability, and other
visible and non-visible differences. All members of this class are
expected to contribute to a respectful, welcoming and inclusive
environment for every other member of the class.

\subsection{Essential Student Information}\label{sec-student-info}

For essential information about student academic success, please see the
\href{https://docs.google.com/document/d/1rXNpNGs1K7nIxcS73o6R-fxZqPIWQwS9gHD7XpIqjhM/edit?tab=t.0#heading=h.apbuhr7p11ak}{SDSU Student Academic Success Handbook}.

SDSU provides disability-related accommodations via Student Disability
Services: \url{https://sds.sdsu.edu} (email:
\href{mailto:sds@sdsu.edu}{\nolinkurl{sds@sdsu.edu}}). Please allow
10--14 business days for processing.

Class rosters are provided to the instructor with the student's legal
name. Please let the instructor know if you prefer an alternate name
and/or gender pronoun.

If you need to be absent from class for a religious observance, please
notify me in writing during the first two weeks of the semester so that
we can make any necessary arrangements.

\subsection{Land Acknowledgement}\label{sec-land}

San Diego State University sits on Kumeyaay land. The Kumeyaay people
have lived in this region for over 10,000 years and continue to live
here today.

\subsection{Your Responsibility}\label{sec-responsibility}

This syllabus constitutes our course contract. You are responsible for
reading and understanding all policies stated here.




\end{document}
